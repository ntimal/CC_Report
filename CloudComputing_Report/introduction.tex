\section{Introduction}
\label{sec:intro}
This document is constructed in response to WantCloud BV�s interest in moving a new application to the cloud. 
To this end the CTO of WantCloud wants us to create a (relatively) simple application to identify what types of trade-offs exists when deploying an application on the cloud.

We chose Amazon  Web Services (AWS)\footnote{http://aws.amazon.com/} as our cloud provider because it is one of the major companies providing resources to external parties to date. 
If WantCloud BV will follow through with their wishes to deploy an application in the cloud most likely AWS will be one to seriously consider. 
As part of AWS�s policy it offers new users of the cloud a free usage tier\footnote{http://aws.amazon.com/free/}. 
For our system we solely used the accommodations provided by the free usage tier, in particular the free use  of Elastic Cloud Computing(EC2) micro instances  and Elastic Load Balancer.        

We created an application which allows users to convert a video file to a desired format using a 
client-server model\footnote{http://en.wikipedia.org/wiki/Client\-server\_model}.  
Although this may seem to be a simple application its workflow can contain very demanding operations. 
That is client(/user) needs to upload a video file to the server, the server in turn processes the uploaded file and converts and returns the converted file to the client. 
The client then stores the converted file to its local file system. 
We therefore find this application to be fitting for assessing the tradeoffs inherent in cloud based applications.

Our cloud based system will implement the following. 
Automation, which means that a servlet will automatically be assigned to a EC2 micro instance. 
Auto-scaling, that is depending on the amount of client requests addition (when CPU usage of existing instances are high) or removal (when CPU usage of an instance is below a low threshold  and can be compensated by an existing instance) EC2 micro instances. 
Load balancing, equally dividing the work between instances using Round Robin scheduling\footnote{http://en.wikipedia.org/wiki/Round\-robin\_scheduling}. 
Reliability is achieved attaching by every micro instance to a persistent storage volume. 
And last but not least monitoring which will provide a comprehensive view of e.g. the amount of resources used by each live micro instance. 
The implementation of the aforementioned features will be accomplished by interfacing AWS with Python�s boto framework. 
The code of this framework can be found at this location\footnote{ http://code.google.com/p/boto}. 
This interface extends among other components of AWS  to  EC2  and Simple Storage Service(S3) which will be used by our cloud based system.      


The rest of the report is organized as follows. Section ~\ref{sec:appl} gives a more in-depth view of the application that is implemented. 
Section ~\ref{sec:design} discusses how the features of the cloud based system are implemented (automation, auto-scaling, load balancing, reliability and monitoring). 
The experimental setup  and the experiments conducted are discussed in Section ~\ref{sec:exp}. 
Any limitations of our cloud based system based on the experimental results from Section ~\ref{sec:exp} appear in Section ~\ref{sec:disc}. 
We will also conclude in Section ~\ref{sec:disc}. This report also contains two appendices. Appendix A gives an overview of the amount of time spend for this project.
Appendix B lists the repositories used for the project.
