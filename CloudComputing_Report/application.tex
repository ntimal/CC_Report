\section{Application}
\label{sec:appl}
We have constructed a video converter tool in Java\footnote{http://www.java.com} which allows an user to specify what file needs to be converted and to what format it will be converted. 
The conversion from one format to another is done in the cloud. 
Using a client-server model, the client uploads  a file and additionally provides the output format. 
Whereas the server, which is provided by the cloud, deals with converting the uploaded file. 
When the conversion is done the converted file is written to the local file system of the client. 
We have used the Jersey\footnote{http://jersey.java.net/}Framework to create a servlet which is constructed using a REST\footnote{http://en.wikipedia.org/wiki/Representational\_state\_transfer} based approach. 
This servlet contains the algorithm for converting a video and is deployed on a Tomcat server in the cloud. 
The algorithm uses several methods from the Xuggle\footnote{http://www.xuggle.com/} framework. 
This framework  is  free and open-source and is used for audio/video manipulation. 
As for the client we used Jersey Client Framework which provides a comprehensive interface for communicating with a server.
\pagebreak[1]  
\subsection{Requirements}
In this paragraph we will elaborate what requirements are essential for our application. 
The requirements can be subdivided into two categories: \emph{functional} and \emph{nonfunctional requirements}. 
A fitting \emph{functional requirement} for our application is that it needs to convert to and from  all commonly known codecs/video formats 
e.g. Windows Media Video(WMV)\footnote{http://en.wikipedia.org/wiki/Windows\_Media\_Video} 
and QuickTime MOV\footnote{http://en.wikipedia.org/wiki/QuickTime}.  
As a \emph{nonfunctional requirement} the servlet needs to be deployed on the cloud, to achieve this goal we will be using the facilities provided by AWS. 
In particular Amazon  Simple Storage (S3) for storing the WAR file created by the Jersey Servlet and Amazon Elastic Compute Cloud(EC2). 
An EC2 instance will be comprised of our servlets computational logic, that is once the servlet is launched on an EC2 instance, the instance will be able to execute a video conversion.
 
